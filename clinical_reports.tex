\section{Clinical evaluation of participants}
\label{app:clinical-eval}

\paragraph{PA1}
\emph{Diagnosis:} 58 y.o., originally left-handed male diagnosed with bulbar-onset
\ac{als} 2 years prior at age 56.
Recently, symptoms started progressing more on the right side.
and resulted in immobility of the left limbs and partial immobility of the right
limbs, but right arm and finger movement were possible, albeit with effort.
\emph{Presentation:} PA1 presented lying reclined in a hospital bed, the
experimental setup was positioned on a movable folding table.
\emph{Communication:} Due to complete anarthria, PA1 communicated using vocal
grunts and a text-to-speech
\ac{aac} tablet operated with the right hand.
\emph{Eye movements:} He reported fatigue linked to his disease progression when
visually fixating for prolonged times, but otherwise reported no eye movement
abnormalities or visual defects.

\paragraph{PB1}
\emph{Diagnosis:} 41 y.o., originally left-handed male diagnosed with \ac{fa}
29 years prior, at age 12.
Symptoms resulted in slowed, effortful body and limb movement and affected
fine motor control, and most severely affected the left side which made him
effectively right handed.
\emph{Presentation:} PB1 presented in an electric wheelchair, which was
positioned in front of a table with the experimental setup.
\emph{Communication:} Verbal communication was possible, but dysarthria resulted
in slowed, slurred speech.
He was able to effectively control a PC and a smartphone, but writing was difficult.
\emph{Eye movements:} PB1 reported impaired visual pursuit of moving targets and
fatigue and discomfort fixating.
Effectively, we noticed effort when making eye movements, which resulted in
large muscular artifacts registered in he \ac{eeg} and \ac{eog}
Additionally, he reported minor nystagmus which we also noticed in the horizontal
\ac{eog}.

\paragraph{PB2}
\emph{Diagnosis:} 43 y.o., right-handed female diagnosed with \ac{fa}
symptom-free at age 9, with symptom onset at age 12, 31 years prior to the
experiment.
Symptoms resulted in slowed, effortful body and limb movement and affected
fine motor control.
\emph{Communication:} Dysarthric, characterized by large effort necessary to speak.
Hand signaling was also often used, but also required effort.
\emph{Eye movements:} PB2 was affected by especially severe horizontal eye
oscillations with onset 10 years prior to the experiment.
Fixating was also fatiguing and uncomfortable, especially when not wearing
glasses.

\paragraph{PB4}
\emph{Diagnosis:} 48 y.o., right-handed male diagnosed with \ac{fa} 30 years
prior at age 18.
\emph{Presentation:} PB4 was not strictly wheelchair-bound.
\emph{Communication:} Speech was relatively fluent, only slightly slurred,
leading to mild dysarthria.
\emph{Eye movements:} Pursuit and fixation were impaired, as characterized by
self-reported fatigue and discomfort. Nystagmus.

\paragraph{PC2}
\emph{Diagnosis:} 43 y.o., originally right-handed male diagnosed with \ac{lis} due to
ischemic brain stem stroke 6 years prior, at age 37. He underwent a tracheostomy.
\emph{Presentation:} PC2 was seated in an electric wheelchair controlled by a
caregiver, and his head was fixated in a headrest.
This headrest contacted the \ac{eeg} cap occipitally and centrotemporally, which
interfered slightly with cap and electrode positioning and optimization.
Additionally, heavy movement artifacts were noticed in the \ac{eeg} due to the
combination of breathing, aspiration, the headrest and maintaining correct
positioning for the experiment.
The \ac{eeg} cap was held in place using a chest strap.
\emph{Communication:} PC2 was completely anarthric and had no preserved facial
movement. He communicated with caregivers by responding to prompted letters and
words with upwards eye movement signaling.
Additionally, he sometimes made use of an emergency button which he could
control with his left hand, the only extremity in which he had some residual
control capability.
\emph{Eye movements:}
Right eye: preserved vertical movement, paralysis of laterality to the right and to the left,
which resulted in the inability to perform horizontal movements.
Left eye: preserved vertical movement, adduction possible, abduction deficit.
Deviation in the left eye resulting in diplopia, corrected by a prism glass.
Self-reported tremors and fixation fatigue and discomfort.

\paragraph{PC3}
\emph{Diagnosis:} 43 y.o., right-handed female diagnosed with hemorrhagic stroke
4 years prior, at age 39. She was tetraplegic and underwent a tracheostomy.
\emph{Presentation:} PC3 was seated in an electric wheelchair
\emph{Communication:} PC3 was completely anarthric, but had sufficient residual
motor control to indicate letters and symbols on a letterboard by pointing with
movements of the right index finger and hand.
\emph{Eye moevements:}
Right eye: complete ptosis
When corrected, downward movement was preserved and incyclotorsion was possible
through preserved cranial nerve IV. Ophthalmoplegia in other directions.
Left eye: Limited of adduction, complete verticality, horizontal nystagmus.
Intermittend nystagmus.
Self reported involuntary saccades and fixation fatigue or discomfort.

\paragraph{PC4}
\emph{Diagnosis:} 54 y.o., originally right-handed male diagnosed with ischemic brainstem
stroke 11 years prior, at age 53. He is tetraplegic and underwent a tracheostomy.
Symptoms most pronounced on left side.
\emph{Presentation:} The tracheostomy resulted in frequent aspiration which
interfered with the experiment and recording.
This and other factors related to the condition and positioning in the wheelchair
resulted in large and frequent muscular and movement artifacts in the \ac{eeg}.
Head bobbing When concentrating on the experimental task.
\emph{Communication:} PC4 had sufficient residual motor control
to indicate letters and symbols on a letterboard by pointing with
movements of the right index finger and hand.
\emph{Eye movement:}
A corneal abcess in the left eye affected it's motility and resulted in deviation.
The left eye had poor vision (only able to
distinguish shapes) and could not be closed.
The right eye was not affected.
The deviation resulted in fatigue and discomfort fixating.
Hypermetropia.
\todo{this is self-reported, official oculomotor diagnosis needed}
